\documentclass[letterpaper,11pt]{article}

\usepackage{latexsym}
\usepackage[empty]{fullpage}
\usepackage{titlesec}
\usepackage{marvosym}
\usepackage[usenames,dvipsnames]{color}
\usepackage{verbatim}
\usepackage{enumitem}
\usepackage[hidelinks]{hyperref}
\usepackage{fancyhdr}
\usepackage[english,portuguese]{babel}
\usepackage{tabularx}
\usepackage{hyphenat}
\usepackage{fontawesome}
\input{glyphtounicode}

\pagestyle{fancy}
\fancyhf{}
\fancyfoot{}
\renewcommand{\headrulewidth}{0pt}
\renewcommand{\footrulewidth}{0pt}

\addtolength{\oddsidemargin}{-0.5in}
\addtolength{\evensidemargin}{-0.5in}
\addtolength{\textwidth}{1in}
\addtolength{\topmargin}{-0.5in}
\addtolength{\textheight}{1.0in}

\urlstyle{same} 
\raggedbottom 
\justifying 
\setlength{\tabcolsep}{0in} 

\titleformat{\section}{
  \vspace{-4pt}\scshape\raggedright\large
}{}{0em}{}[\color{black}\titlerule \vspace{-5pt}]

\pdfgentounicode=1

\newcommand{\resumeItem}[1]{
  \item\small{
    {#1 \vspace{-2pt}}
  }
}

\newcommand{\resumeSubheading}[4]{
  \vspace{-2pt}\item
    \begin{tabular*}{0.97\textwidth}[t]{l@{\extracolsep{\fill}}r}
      \parbox[t]{0.7\textwidth}{\textbf{#1}} & \textit{\small #4}\\
      \textit{\small#3} \\
    \end{tabular*}\vspace{-7pt}
}

\newcommand{\resumeSubSubheading}[2]{
    \vspace{-2pt}\item
    \begin{tabular*}{0.17\textwidth}{l@{\extracolsep{\fill}}r}
      \textit{\small#1} & \textit{\small #2} \\
    \end{tabular*}\vspace{-7pt}
}

\newcommand{\resumeEducationHeading}[4]{
  \vspace{-2pt}\item
    \begin{tabular*}{0.97\textwidth}[t]{l@{\extracolsep{\fill}}r}
      \textbf{#1} & #2 \\
      \textit{\small#3} & \textit{\small #4} \\
    \end{tabular*}\vspace{-5pt}
}

\newcommand{\resumeProjectHeading}[2]{
    \vspace{-2pt}\item
    \begin{tabular*}{0.97\textwidth}{l@{\extracolsep{\fill}}r}
      \small#1 & #2 \\
    \end{tabular*}\vspace{-7pt}
}

\newcommand{\resumeOrganizationHeading}[4]{
  \vspace{-2pt}\item
    \begin{tabular*}{0.97\textwidth}[t]{l@{\extracolsep{\fill}}r}
      \textbf{#1} & \textit{\small #2} \\
      \textit{\small#3}
    \end{tabular*}\vspace{-7pt}
}

\newcommand{\resumeSubItem}[1]{\resumeItem{#1}\vspace{-4pt}}

\renewcommand\labelitemii{$\vcenter{\hbox{\tiny$\bullet$}}$}

\newcommand{\resumeSubHeadingListStart}{\begin{itemize}[leftmargin=0.15in, label={}]}
\newcommand{\resumeSubHeadingListEnd}{\end{itemize}}
\newcommand{\resumeItemListStart}{\begin{itemize}}
\newcommand{\resumeItemListEnd}{\end{itemize}\vspace{-5pt}}

\begin{document}



%----------- CABEÇALHO -----------
\begin{center}
    {\Huge \textbf{Arthur Silva Dantas}} \\[0.5em]
    \small
    \faEnvelope \hspace{.5pt} \href{mailto:contato.arthurdantas.dev@gmail.com}{contato.arthurdantas.dev@gmail.com} \hspace{0.8em}
    \textbar \hspace{0.5em}
    \faLinkedinSquare \hspace{.5pt} \href{https://www.linkedin.com/in/arthur-sd15/}{Arthur Dantas} \hspace{0.8em}
    \textbar \hspace{0.5em}
    \faGithub \hspace{.5pt} \href{https://github.com/Arthur-SD15}{Arthur-SD15} \hspace{0.8em}
    \textbar \hspace{0.5em}
    \faGlobe \hspace{.5pt} \href{http://arthurdantas.com.br/}{arthurdantas.com.br} \\[0.5em]
    \faMapMarker \hspace{.5pt} Campo Grande, MS, Brasil
\end{center}



%----------- RESUMO -----------
\section{Resumo}
\begin{tabular}{p{0.97\textwidth}}
    Engenheiro de Software Full Stack, apaixonado por desenvolver soluções inovadoras e colaborando com projetos que tenham impacto positivo na sociedade. Tenho a capacidade de aplicar conceitos de Engenharia de Software na indústria, indo além do desenvolvimento de software. Além disso, possuo experiência em trabalhar com diversos conceitos e técnicas aplicadas, para desenvolver aplicações escaláveis, eficientes e de alta qualidade. Bem como, conhecimento com diferentes tecnologias no desenvolvimento, como Next.js, React, TypeScript, Node.js, NestJS e Prisma.\medskip
    
    Interesse: \sotag{Engenharia de Software}; \sotag{Full Stack}; \sotag{Metodologia Ágil}; \sotag{Node.js};
\end{tabular}



%----------- ENSINO -----------
\section{Formação Acadêmica}
  \vspace{3pt}
  \resumeSubHeadingListStart
    \resumeEducationHeading
      {Universidade Federal de Mato Grosso do Sul
      }{Campo Grande - Mato Grosso do Sul, Brasil}
      {Bacharelado em Engenharia de Software;   \textbf}{Mar 2024 \textbf{--} Dec 2028 (Previsão de Término)}
    
    \resumeEducationHeading
        {Instituto Federal de Mato Grosso do Sul
      }{Nova Andradina - Mato Grosso do Sul, Brasil}
      {Ensino Médio Profissionalizante em Técnico em Informática;   \textbf}{Fev 2021 \textbf{--} Dec 2023}
  \resumeSubHeadingListEnd



%----------- HABILIDADES -----------
\section{Habilidades}
  \vspace{5pt}
  \resumeSubHeadingListStart
    \small{
        \setlength{\itemsep}{0pt}
        \item \textbf{Linguagens:} Python, JavaScript, TypeScript, SQL
        \item \textbf{Tecnologias:} Node.js, React, Next.js, NestJS, PostgreSQL, Prisma, Git, Postman
        \item \textbf{Metodologias:} Ágil, Scrum, Kanban, OOP
    }
  \resumeSubHeadingListEnd



%----------- EXPERIÊNCIA -----------
\section{Experiência}
  \vspace{3pt}
  \resumeSubHeadingListStart
    \resumeSubheading
      {Feira de Tecnologias, Engenharias e Ciências de Mato Grosso do Sul}{Campo Grande, Brasil}
      {Engenheiro de Software}{Mar 2024 \textbf{--} Dez 2024}
        \resumeItemListStart
            \resumeItem{Atuei como Engenheiro de Software Full Stack, com ações na manutenção e no desenvolvimento de novas funcionalidades de um sistema próprio para a Feira de Tecnologias, Engenharias e Ciências de Mato Grosso do Sul. A referida é uma grande Feira de Ciências do Centro-Oeste Brasileiro. Por atingir um grande número de pessoas, é necessário automatizar tarefas e facilitar processos que exigem pouco tempo.}
            \resumeItem{Com diversas fases, como Inscrição de Projetos, Avaliação Online, Fase de Ajustes, Avaliação Presencial e Emissão de Certificados. Dentro desse processo, existem diferentes roles e regras de negócio para Orientadores, Coorientadores, Avaliadores, Estudantes e Administradores.}
            \resumeItem{Durante o desenvolvimento, foram aplicados conceitos da Engenharia de Software, utilizando metodologia ágil e outros princípios. A aplicação foi desenvolvida com base na Clean Architecture (Arquitetura Limpa) e emprega tecnologias como Node.js, TypeScript, React, NestJS e Prisma.}
        \resumeItemListEnd

    \resumeSubheading
      {Núcleo Interdisciplinar de Pesquisa, Estudo e Desenvolvimento em Tecnologia da Informação}{Nova Andradina, Brasil}
      {Pesquisador}{Ago 2022 \textbf{--} Fev 2024}
        \resumeItemListStart
            \resumeItem{Com o apoio da Fundação de Apoio ao Desenvolvimento do Ensino, Ciência e Tecnologia do Estado de Mato Grosso do Sul, atuei como pesquisador no Núcleo Interdisciplinar de Pesquisa, Estudo e Desenvolvimento em Tecnologia da Informação.}
            \resumeItem{Durante essa primeira e vasta experiência como pesquisador, estive envolvido em uma pesquisa direcionada à inovação e tecnologia, abordando a problemática do alto grau de abstração na modelagem de sistemas por meio de diagramas, especialmente os diagramas da UML. Essa característica dificulta a inclusão de pessoas com deficiência visual na Engenharia de Software, pois os leitores de tela não conseguem interpretar adequadamente essas ferramentas CASE, na modelagem de sistemas.}
            \resumeItem{Diante dessa questão específica da Engenharia de Software, eu como pesquisador, além de me aprofundar em temas sobre a modelagem de sistemas, dediquei-me a resolver essa problemática, desenvolvendo um material didático que torna o processo menos abstrato e mais tangível, promovendo, assim, a inclusão.}
        \resumeItemListEnd

  \resumeSubHeadingListEnd



%----------- PROJETOS -----------
\section{Projetos}
    \vspace{3pt}
    \resumeSubHeadingListStart
      
      \resumeProjectHeading
        {\textbf{ClientServer-RenatoRodas} $|$ \emph{\href{https://github.com/Arthur-SD15/ClientServer-RenatoRodas}{\color{blue}GitHub}}}{}
          \resumeItemListStart
            \resumeItem{O projeto desenvolvido consistiu na criação de uma aplicação web completa de e-commerce para o "Renato Rodas". Utilizando PostgreSQL, foram inseridos 20 produtos e implementados métodos para inserção, remoção e recuperação de itens. A interface foi construída com React.js, Next.js e estilizada com Tailwind. A comunicação entre cliente e servidor foi realizada via API Fetch, com Express e Sequelize para gerenciar as rotas e os métodos do banco de dados. }
          \resumeItemListEnd
      
      \resumeProjectHeading
        {\textbf{Autenticacao-RenatoRodas} $|$ \emph{\href{https://github.com/Arthur-SD15/Autenticacao-RenatoRodas}{\color{blue}GitHub}}}{}
          \resumeItemListStart
            \resumeItem{Este projeto implementa um sistema simulando um e-commerce para o "Renato Rodas" de autenticação de usuários, utilizando um servidor Express.js com JWT (JSON Web Token). O usuário precisa estar cadastrado no banco de dados, e a autenti cação ocorre com a verificação do nome de usuário e senha. Após a autenticação, o servidor gera um token. O sistema também permite registrar novos usuários e verificar se já estão cadastrados. As senhas são armazenadas de forma criptografada utilizando o módulo crypto.}
          \resumeItemListEnd
      
    \resumeSubHeadingListEnd
\end{document}
